%-------definitions-----
\newcommand{\Author}{Kay Hannay <klinux@hannay.de>} 
\newcommand{\Title}{efaLive}
\newcommand{\Keywords}{efaLive efa rowing canoeing}
\newcommand{\DocDate}{12.07.2016}
%--------------------------

\documentclass[a4paper,12pt,twoside]{article}
\usepackage{geometry}
\geometry{a4paper,left=35mm,right=25mm, top=25mm, bottom=20mm}
\usepackage[utf8]{inputenc}
\usepackage[T1]{fontenc}
\usepackage[english,ngerman]{babel}
\usepackage[pdftex]{graphicx}
\pagestyle{headings}
\usepackage{caption}[2008/08/24]
\addto\captionsngerman{\renewcommand\figurename{Abb.}}
\addto\captionsenglish{\renewcommand\figurename{Fig.}}
\renewcommand{\floatpagefraction}{.8}
\usepackage{hyperref}
\setlength{\emergencystretch}{1em}

\usepackage[raggedright]{titlesec}
\titleformat{\paragraph}[hang]{\normalfont\normalsize\bfseries}{\theparagraph}{1em}{}
\titlespacing*{\paragraph}{0pt}{3.25ex plus 1ex minus .2ex}{0.5em}

\hypersetup{%
  pdftitle={\Title},%
  pdfauthor={\Author},%
  pdfkeywords={\Keywords},%
}%
\title{\Title}
\author{\Author}
\date{\DocDate}
\begin{document}
\pagenumbering{roman}
%\clearpage\setcounter{page}{1}

\begin{titlepage}
    \vspace*{1cm}
    \begin{center}
        \begin{figure}
            \centering
            \includegraphics[width=9.745cm,height=7.308cm]{screenshots/efaLivede-img1.png}
        \end{figure}
        \Huge
        \textbf{efaLive} \\[0.1cm]
        \LARGE
        Anleitung \\[5cm]
    \end{center}
    \normalsize
    \vspace*{4cm}
    \textbf{Datum:} {\DocDate} \\
    \textbf{Version:} 1.7 \\
    \textbf{efaLive:} 2.4-x86 \\
    \textbf{Kay Hannay} <klinux@hannay.de> \\
\end{titlepage}

%\clearpage\setcounter{page}{1}
%\bigskip

%\setcounter{tocdepth}{10}
%\renewcommand\contentsname{Inhaltsverzeichnis}
\tableofcontents
%\section{}
\clearpage\setcounter{page}{1}
\pagenumbering{arabic}
\section{Einführung}
\label{sct:einfuehrung}
Diese Anleitung beschreibt, wie man die efaLive-CD benutzt und was man
mit ihr machen kann. Eine Live-CD ist eine CD, von der Computer
gestartet werden können. Das System läuft komplett von der CD und lässt
die Festplatte unangetastet. Daher eignen sich solche Live-CDs
besonders für Demos von Software, Installationen oder auch für
Reparaturen an der Software auf der Festplatte des Computers.

Ein weiteres Merkmal von Live-CDs ist, dass alle Änderungen, die während
des Betriebs vorgenommen wurden, nach dem Herunterfahren des Systems
verloren gehen. Da die Festplatte nicht angetastet wird und eine CD-ROM
nicht beschreibbar ist, gibt es keine Möglichkeit, veränderte Daten
über einen Neustart des Computers hinweg zu speichern. Mit Hilfe eines
USB-Speicher-Sticks kann man dieses Problem jedoch umgehen. Näheres
dazu später.

Das Betriebssystem, welches der CD zugrunde liegt und im Falle einer
Installation auch installiert wird, ist Debian GNU/Linux \cite{DEB1}.
Da es sich bei GNU/Linux um Open-Source-Software handelt, fallen keine
Linzenzkosten an. Aufgrund dieser Tatsache ist es überhaupt möglich,
diese Live-CD anzubieten. Mit Microsoft Windows wäre das aus
rechtlichen Gründen nicht möglich.

Das gleiche gilt für die Software efa \cite{EFA1}, die ebenfalls
Open-Source ist und um die es bei dieser Live-CD geht.

\bigskip
\textbf{ACHTUNG:} Auch wenn die hier beschriebene efaLive-CD die Festplatte des
Computers nicht verändert, so kann doch nicht gänzlich ausgeschlossen
werden, dass das Verhalten des Computers durch die Benutzung der
Live-CD in irgendeiner Weise beeinflusst wird. Ich möchte ausdrücklich
darauf hinweisen, dass die Benutzung der CD auf eigene Gefahr erfolgt
und ich für Schäden, die an Hard- und/oder Software entstehen, nicht
hafte.
\bigskip

\section{efaLive-CD}
\label{sct:efalivecd}
\subsection{Hardwarevoraussetzungen}
\label{sct:live_hardware}
Dies sind die Hardwareanforderungen, wenn man die Live-CD benutzen
möchte. Die Angaben sind als Minimalvoraussetzung zu verstehen.

\begin{itemize}
    \item Intel Pentium III Prozessor mit 600MHz
    \item 256 MB Arbeitsspeicher
    \item CD-ROM Laufwerk oder USB Anschluss
    \item Monitor mit einer Auflösung von 1024x768 Pixeln
    \item ggf. USB Anschluss für die Datensicherung
\end{itemize}

\subsection{efaLive-CD erstellen}
\label{sct:cd_erstellen}
efaLive kann im Internet unter \cite{EFA4} als ISO CD Abbild
heruntergeladen werden. Um aus dem CD Abbild eine CD zu erstellen,
bieten fast alle Brennprogramme einen Menüpunkt "`Abbild brennen"', 
"`Write CD image"' oder ähnlich. Für weitere Informationen bitte 
die Dokumentation des entsprechenden Programms konsultieren. 

Alternativ kann das Abbild auch auf einen USB Stick kopiert werden.

\subsection{Oder per USB Stick}
\label{sct:usb_stick}
efaLive kann auch auf einen USB Stick kopiert werden.
Der Computer muss allerdings in der Lage sein, von einem USB Stick zu
starten, was gerade bei älteren Computern nicht unbedingt der Fall ist.

Unter Linux kann das ISO Abbild mit dem Befehl \texttt{dd if={\textless}EFA\_LIVE\_ISO{\textgreater} 
of=/dev/sdb} (wenn /dev/sdb der USB Stick ist) kopiert werden. Achtung: 
wenn für \texttt{of=} das falsche Gerät ausgewählt wird, werden womöglich 
ungewollt Daten gelöscht!

Unter Windows kann das Programm Win32 Disk Imager \cite{IMG1} verwendet
werden. Da dieses Programm nur Dateien mit der Endung \texttt{.img} 
verarbeitet, muss das ISO Abbild entweder umbenannt werden oder in dem 
Feld "`Dateiname"' des Programms \texttt{*.iso} eingegeben werden.

Ansonsten gelten die gleichen Bedingungen, wie bei einer CD.


\subsection{efaLive ausführen}
\label{sct:live_ausfuehren}
Der Start von efaLive ist ganz einfach:

\begin{enumerate}
    \item CD in das CD-ROM Laufwerk des Computers einlegen oder USB Stick
        einstecken
    \item Computer (neu) starten
    \item In dem Auswahl-Bildschirm (Abb.~\ref{fig:syslinux}) den Punkt
        "`efaLive"' (Deutsch oder Englisch) auswählen (\texttt{{\textless}Enter{\textgreater}} 
        drücken)
\end{enumerate}

\begin{figure}
    \centering
    \fbox{\includegraphics[width=10cm]{screenshots/efaLivede-img2.png}}
    \caption{Auswahlbildschirm Bootloader (Syslinux)}
    \label{fig:syslinux}
\end{figure}

Bei dem Start wird automatisch efa mit der Bootshausoberfläche geöffnet.
Weitere Informationen zur Benutzung von efa sind in der Dokumentation
zu efa zu finden \cite{EFA2}.

Nach einer Weile sollte ein Fenster wie in Abb.~\ref{fig:efalivesetup_live} zu
sehen auf dem Bildschirm erscheinen. Hier können einige Einstellungen vorgenommen
werden. In der Regel können diese vorerst so bleiben, wie sie sind. 
Wenn alle Einstellungen gemacht sind, auf "`Ok"' klicken.

Das Fenster efaLive Setup kann, wenn efa läuft, jederzeit durch drücken
der Tastenkombination \texttt{{\textless}Strg{\textgreater}+{\textless}F12{\textgreater}} 
aufgerufen
werden, um die Einstellung wieder zu ändern. Vor dem Öffnen von efaLive
Setup muss das Passwort des Benutzers "`efa"' eingegeben werden.

\begin{figure}
    \centering
    \includegraphics[width=8cm]{screenshots/efalive_setup.png}
    \caption{efaLive Setup}
    \label{fig:efalivesetup_live}
\end{figure}

Das Standard-Passwort vom Benutzer "`root"' ist "`livecd"', das von
"`efa"' ist "`efalive"'.


\subsection{Daten sichern im Live-Betrieb}
\label{sct:live_sichern}
Um Änderungen abzuspeichern, kann man einen USB Stick nutzen. Dazu
einfach einen beliebigen USB Stick nehmen und die Datei \texttt{persistence.zip}
aus dem Verzeichnis \texttt{snapshot} der CD auf diesen entpacken, so dass im
Hauptverzeichnis des Sticks die Datei "`\texttt{persistence}"' liegt. Den Stick nun
vor dem Start des Computers in einen freien USB Steckplatz stecken und
danach mit der efaLive-CD starten. Während des Startvorgangs sollte der
Stick vom System erkannt und automatisch zum Abspeichern von
sogenannten Snapshots genutzt werden. Dies bedeutet, dass der Inhalt
der Datei \texttt{persistence} beim Start des Computers in das Heimatverzeichnis
des Benutzers "`efa"' kopiert wird. In dem
Heimatverzeichnis befinden sich die Konfigurations- und Benutzerdaten
von efa.

Sobald der Computer heruntergefahren wird, wird der Inhalt des
Heimatverzeichnisses wieder in die Datei persistence kopiert. Es findet
also nur beim Start und beim Herunterfahren des Computers eine
Kopieraktion statt. Daher sollte man den USB Stick erst wieder
entfernen, wenn der Computer vollständig heruntergefahren wurde.


\section{Installation}
\label{sct:installation}
\subsection{Hardwarevoraussetzungen}
\label{sct:inst_hardware}
Auch hier gilt, dass sich die Angaben als Minimalanforderungen
verstehen.

\begin{itemize}
    \item Intel Pentium III Prozessor mit 600MHz
    \item 256 MB Arbeitsspeicher
    \item 2 GB Festplatte
    \item CD-ROM Laufwerk oder USB Anschluss (nur für Installation)
    \item Monitor mit einer Auflösung von 1024x768 Pixeln
    \item ggf. USB Anschluss für die Datensicherung
\end{itemize}

Der Arbeitsspeicher kann evtl. noch geringer gewählt werden. In diesem
Fall ist jedoch keine grafische Installation mehr möglich. Die
Installation im Text-Modus ist zwar auch nicht sehr schwierig, wird
jedoch in diesem Dokument nicht betrachtet.

Vor der Installation ist es sinnvoll, sich bereits darüber Gedanken zu
machen, welche Hardwarekomponenten zum Betrieb des Systems wirklich
benötigt werden. Eine Anregung gibt Kapitel~\ref{sct:peripherie}.


\subsection{Die Installationsschritte}
\label{sct:inst_schritte}

\begin{minipage}{\linewidth}
    \centering
    \captionsetup{type=figure}
    \includegraphics[width=10cm]{screenshots/select_language.png}
    \captionof{figure}{Auswahl der Sprache}
    \label{fig:inst_sprache}
\end{minipage}

\bigskip
In dieser Beschreibung kann ich leider nur auf bestimmte Aspekte der
Installation eingehen. Ich gehe davon aus, dass ein einfacher
Desktop-PC mit einer Festplatte, einem CD-Rom Laufwerk und optional
einer Netzwerkkarte für kabelgebundene Netzwerke zum Einsatz kommt.
Ferner nehme ich an, dass die komplette Festplatte des Systems gelöscht
werden kann. Weitere (allgemeinere) Informationen zur Installation von
Debian GNU/Linux gibt es unter \cite{DEB2}.

\textbf{ACHTUNG:} Bei der Installation nach dieser Anleitung wird die gesamte
Festplatte des Computers gelöscht! Es gehen also alle auf der
Festplatte gespeicherten Daten verloren! Es ist möglich, dieses
Verhalten zu beeinflussen, jedoch wird darauf in dieser Anleitung nicht
näher eingegangen.

Im ersten Schritt muss die Sprache gewählt werden. Danach auf den Knopf
"`Continue"' klicken. Die Auswahl des
Standortes und der Tastatur kann in den meisten Fällen ohne weitere
Aktion mit einem Klick auf den Knopf
"`Weiter"' bestätigt werden.


\subsubsection{Computer ohne Netzwerkkarte}
\label{sct:inst_ohne_netzwerk}

\begin{minipage}{\linewidth}
  \centering
  \captionsetup{type=figure}
  \includegraphics[width=10cm]{screenshots/select_network_card.png}
  \captionof{figure}{Auswahl Netzwerkkarte}
  \label{fig:inst_netzwerkkarte}
\end{minipage}

Da heutige Computer meistens eine Netzwerkkarte besitzen, versucht das
Installationsprogramm recht energisch, eine solche zu finden. Wähle den
Punkt "`keine Netzwerkkarte"' aus.

Es folgt ein Warnhinweis, dass keine Netzwerkkarte gefunden wurde. Diese
kann bestätigt werden. Der nächste Abschnitt kann nun übersprungen
werden.

\bigskip
\begin{minipage}{\linewidth}
  \centering
  \captionsetup{type=figure}
  \includegraphics[width=10cm]{screenshots/message_no_network.png}
  \captionof{figure}{Bestätigung Netzwerkkarte}
  \label{fig:inst_best_netzwerkkarte}
\end{minipage}


\subsubsection{Computer mit Netzwerkkarte}
\label{sct:inst_mit_netzwerk}
In der Regel gibt es in den üblichen Netzwerken mit DSL Anschluss oder
ähnlichem auch einen DHCP Server. Dieser Konfiguriert die Netzwerkkarte
automatisch mit den nötigen Einstellungen für das Netzwerk. Meist läuft
ein solcher DHCP Server auf dem Router. Falls die automatische
Konfiguration nicht gelingt, erscheint ein Hinweis, wie in
Abb.~\ref{fig:dhcp_fehler} zu sehen. Andernfalls wird die Konfiguration
automatisch erledigt und es kann mit der Einrichtung der Festplatte
fortgefahren werden.

\begin{minipage}{\textwidth}
    \centering
    \captionsetup{type=figure}
    \includegraphics[width=10cm]{screenshots/message_dhcp_failed.png}
    \captionof{figure}{Netzwerk konnte nicht per DHCP konfiguriert werden}
    \label{fig:dhcp_fehler}
\end{minipage}
\bigskip

Dieser Hinweis kann bestätigt werden. Im folgenden Fenster muss nun
ausgewählt werden, wie das Netzwerk zu konfigurieren ist. Ist die
automatische Konfiguration per DHCP fehlgeschlagen, obwohl in dem
Netzwerk ein DHCP Server existiert, muss man sich nun auf die
Fehlersuche begeben und die automatische Konfiguration wiederholen.

Eine andere Möglichkeit ist, das Netzwerk unkonfiguriert zu lassen, was
ich jedoch nicht für sehr sinnvoll erachte, da man in diesem Fall die
Netzwerkkarte besser gleich ausbauen oder deaktivieren sollte (siehe
Kapitel~\ref{sct:peripherie}).

\begin{minipage}{\textwidth}
    \centering
    \captionsetup{type=figure}
    \includegraphics[width=10cm]{screenshots/select_network_manually.png}
    \captionof{figure}{Manuelle Konfiguration des Netzwerks}
    \label{fig:netzwerk_manuell}
\end{minipage}
\bigskip

In der Regel wird man nun das Netzwerk manuell einrichten wollen. Dazu
den entsprechenden Eintrag wie in Abb.~\ref{fig:netzwerk_manuell} auswählen und
mit \texttt{{\textless}Enter{\textgreater}} bestätigen. In den nun folgenden
Fenstern werden nacheinander die IP Adresse für den Computer, die
Netzmaske, das Gateway und der Nameserver (DNS) für das Netzwerk
abgefragt. Diese Daten bitte unbedingt gewissenhaft eingeben. Eine
falsche Konfiguration kann das gesamte Netzwerk zum Erliegen bringen.


\subsubsection{Konfiguration der Zeitzone}
\label{sct:inst_timezone}

Abhängig vom vorher ausgewählten Land, kann ein Dialog zur Auswahl einer
Zeitzone angezeigt werden.

\bigskip
\begin{minipage}{\linewidth}
    \centering
    \captionsetup{type=figure}
    \includegraphics[width=10cm]{screenshots/select_timezone.png}
    \captionof{figure}{Auswahl Zeitzone}
    \label{fig:inst_timezone}
\end{minipage}


\subsubsection{Festplatte einrichten}
\label{sct:inst_festplatte}

\begin{minipage}{\linewidth}
    \centering
    \captionsetup{type=figure}
    \includegraphics[width=10cm]{screenshots/select_partitioning.png}
    \captionof{figure}{Auswahl Partitionierung}
    \label{fig:partitionierung}
\end{minipage}
\bigskip

Im nächsten Schritt wird die Festplatte partitioniert. Das bedeutet,
dass die Festplatte passend für die Benutzung von efa aufgeteilt wird.
Ich gehe hier davon aus, dass die gesamte Festplatte verwendet werden
soll. Dadurch werden alle Daten, die sich auf der Festplatte befinden,
gelöscht! Über den Eintrag "`Manuell"' kann
dieses Verhalten beeinflusst werden. Allerdings sollte man sich mit der
Partitionierung von Festplatten auskennen, da sonst ungewollt Daten
verloren gehen können.

\begin{minipage}{\linewidth}
    \centering
    \captionsetup{type=figure}
    \includegraphics[width=10cm]{screenshots/select_drive.png}
    \captionof{figure}{Auswahl Festplatte}
    \label{fig:auswahl_festplatte}
\end{minipage}

Als nächstes muss eine Festplatte ausgewählt werden. In diesem Beispiel
befindet sich nur eine Festplatte im Computer, daher kann dieser
Schritt bestätigt werden.

\begin{minipage}{\linewidth}
    \centering
    \captionsetup{type=figure}
    \includegraphics[width=10cm]{screenshots/accept_partitioning.png}
    \captionof{figure}{Partitionierung bestätigen}
    \label{fig:best_partitionierung}
\end{minipage}

Es folgt eine Übersicht, wie die Festplatte aufgeteilt werden soll. Mit
"`Partitionierung beenden und Änderungen übernehmen"' geht es weiter.

\begin{minipage}{\linewidth}
    \centering
    \captionsetup{type=figure}
    \includegraphics[width=10cm]{screenshots/confirm_partitioning.png}
    \captionof{figure}{Sicherheitsabfrage}
    \label{fig:abfrage_partitionierung}
\end{minipage}

An dieser Stelle erfolgt noch einmal eine Warnung, dass alle Daten auf
der Festplatte gelöscht werden, wenn dieser Bildschirm mit
"`Ja"' bestätigt wird.

\subsubsection{Paketverwaltung}
\label{sct:paketverwaltung}

\begin{minipage}{\linewidth}
    \centering
    \captionsetup{type=figure}
    \includegraphics[width=10cm]{screenshots/use_mirror.png}
    \captionof{figure}{Abfrage, ob Spiegelserver genutzt werden soll}
    \label{fig:abfrage_mirror}
\end{minipage}
\bigskip

Das Linux System, welches efaLive zugrunde liegt, kann über das Internet
aktualisiert werden. An dieser Stelle kann nun ein sogenannter
Spiegelserver konfiguriert werden. Verfügt der Computer nicht über eine
Internetverbindung, kann über die Auswahl von
"`Nein"' ohne Spiegelserver fortgefahren
werden. Wenn der Computer jedoch über eine Internetverbindung verfügt,
empfehle ich, einen solchen Server einzurichten.

\begin{minipage}{\linewidth}
    \centering
    \captionsetup{type=figure}
    \includegraphics[width=10cm]{screenshots/select_mirror_country.png}
    \captionof{figure}{Region für Spiegelserver}
    \label{fig:mirror_region}
\end{minipage}

Spiegelserver sind Server im Internet, die sich in regelmäßigen
Abständen mit dem zentralen Paket-Server des Debian Projekts
abgleichen. Dies dient dazu, den zentralen Server nicht zu überlasten.
Es ist sinnvoll, einen Spiegelserver in der Nähe des eigenen Standortes
auszuwählen, da so tendenziell eine höhere
Datenübertragungsgeschwindigkeit zur Verfügung steht.

Nach einem Klick auf "`Weiter"' wird
ein Dialog wie in Abb.~\ref{fig:auswahl_mirror} angezeigt. Hier kann ein
konkreter Spiegelserver ausgewählt werden. Oft kann man an dem Namen
schon erkennen, ob dieser nah am eigenen Standort steht.

\begin{minipage}{\linewidth}
    \centering
    \captionsetup{type=figure}
    \includegraphics[width=10cm]{screenshots/select_mirror.png}
    \captionof{figure}{Spiegelserver auswählen}
    \label{fig:auswahl_mirror}
\end{minipage}
 
Es folgt die Abfrage eines Proxy Servers. In den meisten Fällen wird es
keinen geben. Dann kann das Feld leer gelassen werden.

\begin{minipage}{\linewidth}
    \centering
    \captionsetup{type=figure}
    \includegraphics[width=10cm]{screenshots/set_mirror_proxy.png}
    \captionof{figure}{Proxy Server angeben}
    \label{fig:auswahl_proxy}
\end{minipage}


\subsubsection{Bootloader}
\label{sct:bootloader}

\begin{minipage}{\linewidth}
    \centering
    \captionsetup{type=figure}
    \includegraphics[width=10cm]{screenshots/select_bootloader_mbr.png}
    \captionof{figure}{Bootloader Grub installieren}
    \label{fig:inst_grub}
\end{minipage}
\bigskip

Nun ist es fast geschafft. Das Installationsprogramm fragt nach, wo der
Bootloader installiert werden soll. Was ein Bootloader ist, spielt an
dieser Stelle keine besondere Rolle. Er sollte jedoch in der Regel im
"`Master Boot Record"' installiert werden, da
der Computer das efaLive System sonst nicht automatisch starten wird.
Diesen Dialog also mit "`Ja"' bestätigen.

\begin{minipage}{\linewidth}
    \centering
    \captionsetup{type=figure}
    \includegraphics[width=10cm]{screenshots/finish_installation.png}
    \captionof{figure}{Abschluss}
    \label{fig:abschluss}
\end{minipage}
\bigskip

Es ist geschafft. Die Installation ist vollendet. Wenn Du hier auf
"`Weiter"' klickst, wird der Computer neu
gestartet. Damit nicht wieder von der Live-CD gestartet wird, empfiehlt
es sich, diese nun aus dem CD-ROM Laufwerk zu nehmen, bzw. den USB
Stick zu entfernen.

Wenn nun von der Festplatte gestartet wird, erscheint nicht mehr der
Bildschirm wie in Abb.~\ref{fig:syslinux}. Der gerade installierte
Bootloader tritt nur noch durch eine kurz angezeigte Textzeile in
Erscheinung. An dieser Stelle kann man die \texttt{{\textless}ESC{\textgreater}}
Taste drücken, um in das Menü des Bootloaders zu gelangen. Will man
einen der Einträge in dem Menü editieren, muss man sich
authentifizieren. Die Standardeinstellung ist für den Benutzer
"`root"' und für das Passwort "`livecd"'. 

Drückt man keine Taste, startet efaLive. Wie auch beim Start als
Live-CD, kann das Bild, welches während des Starts angezeigt wird, mit
der Taste \texttt{{\textless}ESC{\textgreater}} geschlossen werden, um Meldungen
des Systems angezeigt zu bekommen.

Ich empfehle dringend, Kapitel~\ref{sct:system_absichern} zu studieren, um das System etwas
sicherer zu machen.

\bigskip
Informationen zur Benutzung von efa gibt es unter \cite{EFA2}.

\bigskip
Viel Spaß mit efaLive!


\section{Administration des Systems}
\label{sct:administration}
\subsection{lokaler Zugang}
\label{sct:lokaler_zugang}
Zur Wartung des Systems wird efaLive-Setup
oder eine Konsole verwendet. Einige Wartungsaufgaben können als
Benutzer "`efa"' durchgeführt werden, andere
nur als Benutzer "`root"'.

Linux Systeme verfügen über einen Zugang für Administrationsaufgaben.
Der zugehörige Benutzer heißt "`root"'.
Meldet man sich als Benutzer "`root"' an,
kann man alles an dem System verändern und auch zerstören. Daher sollte
man wirklich nur für Aufgaben, die solche Rechte erfordern, als
Benutzer "`root"' arbeiten.

Wenn im folgenden davon gesprochen wird, dass Du dich als
"`root"' oder "`efa"' einloggen sollst, dann meint das,
dass Du entweder auf eine der Textkonsolen (Kap.
~\ref{sct:textkonsole}) wechseln oder die Konsole der
Toolbox (Kap.~\ref{sct:toolbox}) verwenden sollst.

Wenn alle Arbeiten erledigt sind, kann man sich mit dem Befehl
\texttt{exit} (nach Verwendung von
\texttt{su -} 2 Mal) wieder ausloggen. Die Konsole
der Toolbox kann auch per Klick auf das X in der rechten oberen Ecke
des Fensters geschlossen werden.

Diesen Schritt bitte nicht vergessen, denn ansonsten kann ein findiger
Mensch das System ganz leicht manipulieren oder gar löschen.
Gegebenenfalls noch mal mit
\texttt{{\textless}Alt{\textgreater}+{\textless}Tab{\textgreater}} nachsehen, ob
noch ein Fenster im Hintergrund geöffnet ist.


\subsubsection{Toolbox}
\label{sct:toolbox}
Die "`Toolbox"' des efaLive-Setup ist die
bevorzugte Variante eine Konsole aufzurufen. In diesem Fall ist man als
Benutzer "`efa"' angemeldet, dessen Passwort
man vor dem Start von efaLive-Setup eingeben muss. Um sich als Benutzer
"`root"' anzumelden, muss der Befehl
\texttt{su -} eingegeben werden. Nun erfolgt die
Abfrage des "`root"' Passwortes.

Das efaLive-Setup kann jederzeit aus efa heraus über
\texttt{{\textless}Strg{\textgreater}+{\textless}F12{\textgreater}} gestartet
werden. Weitere Informationen zu efaLive-Setup gibt es in Kap.
\ref{sct:efalivesetup}.


\subsubsection{Textkonsole}
\label{sct:textkonsole}
Bei Linux Systemen kann man während
des Betriebs von der grafischen Oberfläche auf Textkonsolen wechseln.
Dies geschieht mit den Tastenkombinationen
\texttt{{\textless}Strg{\textgreater}+{\textless}Alt{\textgreater}+{\textless}F1{\textgreater}}
bis \texttt{{\textless}F6{\textgreater}}. Hinter jeder dieser
Tastenkombinationen verbirgt sich eine Textkonsole, auf der man sich
über einen Benutzernamen und ein Passwort anmelden kann. Um wieder zu
der grafischen Oberfläche zu gelangen, muss man die Tastenkombination
\texttt{{\textless}Strg{\textgreater}+{\textless}Alt{\textgreater}+{\textless}F7{\textgreater}}
drücken.

Um sich als Benutzer "`root"' anzumelden kann
man z.B. mit
\texttt{{\textless}Strg{\textgreater}+{\textless}Alt{\textgreater}+{\textless}F1{\textgreater}}
auf eine Textkonsole wechseln und dort bei
\texttt{login} "`root"'
eingeben (und mit \texttt{{\textless}Enter{\textgreater}} bestätigen). Darauf
folgt die Abfrage des Passwortes. Hier ist zu beachten, dass bei der
Eingabe des Passwortes keine Ausgaben auf dem Bildschirm erfolgen. Es
werden also keine Punkte oder Sternchen als Bestätigung der Eingaben
ausgegeben. Bitte nicht verwirren lassen, wenn die Tastatur bis hierher
funktioniert hat, sollte die Eingabe des Passwortes einwandfrei
klappen.

Die Textkonsolen nutzen im Live-Betrieb eine englische Tastatur, daher
ist die Bedienung hier unter Umständen etwas umständlich.


\subsection{Zugang über Netzwerk}
\label{sct:zugang_netzwerk}
Das efaLive System bringt einen SSH Server mit. SSH ist ein Protokoll,
welches es ermöglicht, sich über das Netzwerk an einem entfernten
Computer anzumelden. Auf Computern mit Linux als Betriebssystem ist die
SSH Client Software normalerweise bereits installiert. Für Windows gibt
es z.B. das Programm Putty \cite{PUT1}.

Unter Linux würde z.B. ein Befehl wie \texttt{ssh efa@efalive.efa.local} 
reichen, um eine Konsole auf dem
efaLive Computer zu erhalten, wie sie auch in Kapitel
\ref{sct:textkonsole} erwähnt ist. In einem kleinen Heimnetzwerk
muss vielleicht nach dem \texttt{@} die IP Adresse
des Rechners verwendet werden, statt des Namens. Aus Sicherheitsgründen
ist es nicht erlaubt, sich direkt als
"`root"' Benutzer über SSH anzumelden, daher
habe ich in dem Beispiel den Benutzer "`efa"'
verwendet. Um sich als Benutzer "`root"'
anzumelden, muss wie bei der Verwendung der Toolbox (Kapitel
~\ref{sct:toolbox}) der Befehlt \texttt{su -} verwendet werden.

Für den Zugriff über das Internet muss in dem verwendeten DSL Router
o.ä. vermutlich eine sogenannte Portweiterleitung eingerichtet werden.
Dabei wird ein beliebiger Netzwerk-Port, z.B. 1234, auf den Port 22 des
efaLive Computers umgeleitet (also dessen Netzwerknamen bzw. IP
Adresse). Aus Sicherheitsgründen sollte der Port auf dem Router (im
Beispiel 1234), der auf den efaLive Computer umgeleitet wird, nicht 22
sein, da dies der Standard-Port für den SSH Dienst ist und hier viele
Angriffsversuche aus dem Internet erfolgen.

Wenn der Zugriff auf das efaLive System vom Internet aus möglich ist,
ist es noch wichtiger, für den Benutzer
"`efa"' ein sicheres Passwort zu wählen! Es
sollte möglichst lang sein und Groß-, Kleinbuchstaben, Zahlen und
Sonderzeichen enthalten.


\subsection{Datensicherung}
\label{sct:datensicherung}
\subsubsection{Daten sichern}
\label{sct:daten_sichern}
Das System kann über efaLive-Setup
so eingerichtet werden, dass immer automatisch eine Datensicherung
durchgeführt wird, sobald ein USB Stick in den Computer gesteckt wird. 

Hat die automatische Sicherung funktioniert, ertönen in der Regel drei
kurze Töne. Geht etwas schief, werden 5 lange Töne ausgegeben. In einem
solchen Fall kann man den USB Stick eingesteckt lassen und über den
Dialog "`Speichermedien"' der Toolbox eine
Datensicherung durchführen. Fehlermeldungen können hier am Ende über
"`Details"' eingesehen werden.

Wurde die erfolgreiche Sicherung durch drei kurze Töne bestätigt, kann
man den Stick herausziehen, er wird nach der Sicherung automatisch
ausgehängt.

Besitzt der PC keinen eingebauten Lautsprecher oder funktionieren die
Tonsignale aus einem anderen Grund nicht, kann in efaLive-Setup ein
Dialog eingeschaltet werden, der nach Beendigung der Datensicherung
angezeigt wird (siehe Kapitel~\ref{sct:efalivesetup}).

Es sollte sich nun ein Verzeichnis mit dem Namen\\
\texttt{efaLive\_backup\_YYYYMMDD\_HHMMSS} auf dem
Stick befinden. Wobei \texttt{YYYYMMDD} das aktuelle Datum ist, also z.B.
\texttt{20100228}, und \texttt{HHMMSS} die Uhrzeit, z.B. 134421. In diesem Verzeichnis
befinden sich zwei Dateien,
\texttt{efa\_backup\_YYYYMMDD\_HHMMSS.zip} und
\texttt{efaLive\_backup\_YYYYMMDD\_HHMMSS.zip}.
Erstere Datei enthält die Datensicherung von efa, die Zweite die
Sicherung der efaLive Einstellungen.

Von efaLive werden lediglich einige Einstellungen aus efaLive-Setup
gesichert. Andere Veränderungen am System müssen separat gesichert
werden. Dies ist ein Kompromiss, um möglichst viele Einstellungen zu
speichern, die Datensicherung aber nicht unnötig groß werden zu lassen.

Eine weitere Möglichkeit, eine Datensicherung durchzuführen, ist die
Toolbox, siehe dazu Kapitel~\ref{sct:dialog_speichermedien}. Außerdem kann der Befehl
\texttt{efalive-backup /media/{\textless}MOUNT POINT{\textgreater}} 
als Benutzer "`efa"' benutzt werden. Der Platzhalter "`{\textless}MOUNT 
POINT{\textgreater}"' muss durch den Einhängepunkt des USB Sticks 
ersetzt werden.


\subsubsection{Wiederherstellen}
\label{sct:daten_wiederherstellen}
Der einfachste Weg, eine Datensicherung zurück zuspielen, ist der
Speichermedien-Dialog (Kapitel~\ref{sct:dialog_speichermedien}). Hier
kann eine Sicherung direkt von einem USB-Stick wieder eingespielt
werden. Des weiteren kann der Datensicherungs-Dialog (Kapitel
~\ref{sct:dialog_datensicherung}) verwendet werden.

Schließlich kann auch der Schritt der Wiederherstellung an der
Eingabeaufforderung durchgeführt werden. Dazu muss man sich als
Benutzer "`efa"' einloggen und den Befehl
\texttt{efalive-restore {\textless}DATENSICHERUNG{\textgreater}} eingeben. Der
Name der Sicherungsdatei \texttt{{\textless}DATENSICHERUNG{\textgreater}} ist
eine der beiden Sicherungsdateien mit der Endung \texttt{.zip}. Beide Dateien
einer kompletten efaLive Datensicherung müssen in einem Verzeichnis liegen. 
Alternativ kann auch eine reine efa oder efaLive Datensicherung zurückgespielt
werden. In diesem Fall wird eine Warnmeldung ausgegeben, dass keine vollständige
efaLive Datensicherung zurückgespielt wurde.

Wurde eine efa Datensicherung aus einem nicht-efaLive System eingespielt, muss
der efaLive-Admin in efa angelegt werden. Dies geschieht im Admin-Modus von efa
unter "`Administratoren"' - "`Admin efaLive reparieren"'.

Der Computer sollte nach der Wiederherstellung neu gestartet werden,
damit efa die neuen Daten benutzt. Dies kann über efaLive-Setup
erledigt werden.


\section{efaLive-Setup}
\label{sct:efalivesetup}
efaLive-Setup ist ein Programm, mit
dessen Hilfe verschiedene Einstellungen am efaLive System vorgenommen
werden können. Außerdem enthält efaLive-Setup die
"`Toolbox"', die verschiedene Werkzeuge zur
Verwaltung des efaLive Systems bereitstellt.

\begin{figure}
    \centering
    \includegraphics[width=8cm]{screenshots/efalive_setup.png}
    \caption{efaLive Setup}
    \label{fig:efalivesetup}
\end{figure}

\subsection{Werkzeuge}
\label{sct:efalivesetup_tools}


\subsubsection{Kommandozeile}
\label{sct:kommandozeile}
Die Kommandozeile, oder auch Textkonsole, kann zu verschiedenen, hier im
Dokument beschriebenen, Aufgaben verwendet werden. Nach dem Start
arbeitet man als Benutzer "`efa"'. Sind
Aufgaben als Benutzer "`root"' zu erledigen,
kann man sich über den Befehlt \texttt{su -} als
Benutzer "`root"' einloggen.


\subsubsection{Dateimanager}
\label{sct:dateimanager}
Hinter diesem Knopf verbirgt sich ein einfacher Dateimanager. Hier
können Dateien und Verzeichnisse kopiert, verschoben, gelöscht werden,
Dateien mit dem Editor geöffnet werden und vieles mehr.


\subsubsection{Speichermedien}
\label{sct:dialog_speichermedien}
Will man einen USB-Stick einbinden
oder eine Datensicherung/-wiederherstellung durchführen, kann man
dieses Werkzeug verwenden.

\begin{figure}
    \centering
    \includegraphics[width=5cm]{screenshots/efaLivede-img20.png}
    \caption{Speichermedien Werkzeug}
    \label{fig:dialog_speichermedien}
\end{figure}

Es werden untereinander alle gefundenen Speichermedien angezeigt. Dazu
gibt es jeweils drei Knöpfe. Einen zum Ein- und Aushängen des
entsprechenden Mediums. Der Zweite dient dazu, eine Datensicherung auf
dem entsprechenden Speichermedium durchzuführen. Mit dem dritten Knopf
kann eine Datensicherung wiederhergestellt werden. Dazu wird ein Dialog
geöffnet, in dem man eine Sicherungsdatei auswählen muss. Mehr dazu unter
\ref{sct:daten_wiederherstellen}.

Bewegt man den Mauszeiger über den Namen eines Speichermediums und lässt
den Zeiger dort eine Weile verweilen, werden verschiedene Informationen
zu dem Medium angezeigt, unter anderem der Einhängepunkt. Das ist das
Verzeichnis im Dateisystem, wo das Speichermedium nach dem Einhängen zu
finden ist.

Unter Gerät ist der Gerätename zu finden, über den man das
Speichermedium ansprechen kann.


\subsubsection{Editor}
\label{sct:gui_editor}
Dieser Knopf startet einen einfachen
grafischen Texteditor namens "`Leafpad"', mit
dem z.B. Konfigurationsdateien editiert werden können. (Siehe auch
Kapitel~\ref{sct:editor})


\subsubsection{Datensicherung}
\label{sct:dialog_datensicherung}
Mit diesem Werkzeug können
Datensicherungen erstellt oder wiederhergestellt werden. Der
Mechanismus ist ganz ähnlich zu dem im Speichermedien-Dialog. Nur
können hier Datensicherungen an einer beliebigen Stelle im Dateisystem
erstellt werden.

Nach einem Klick auf den Knopf
"`Datensicherung"' öffnet sich ein Dialog, in
dem ein Verzeichnis ausgewählt werden muss. In dieses Verzeichnis wird
die Datensicherung gespeichert. Das Ende der Datensicherung wird durch
einen Dialog angezeigt, der auch mitteilt, ob die Sicherung erfolgreich
war.

\bigskip
\begin{minipage}{\linewidth}
    \centering
    \captionsetup{type=figure}
    \includegraphics[width=3cm]{screenshots/efaLivede-img21.png}
    \captionof{figure}{Datensicherung}
    \label{fig:dialog_datensicherung}
\end{minipage}
\bigskip

Klickt man auf "`Wiederherstellen"', öffnet
sich ein Dialog, in dem eine Sicherungsdatei (\texttt{.zip})
ausgewählt werden muss. Mehr dazu unter \ref{sct:daten_wiederherstellen}.


\subsubsection{Logdateien}
\label{sct:logfiles}
Mit diesem Knopf kann eine ZIP-Datei erstellt werden, die die wichtigsten
Logdateien des Systems enthält. In einem Dialog kann das Zielverzeichnis 
für das Logdateienpaket gewählt werden.


\subsection{Aktionen}
\label{sct:aktionen}

\begin{minipage}{\linewidth}
    \centering
    \captionsetup{type=figure}
    \includegraphics[width=8cm]{screenshots/efalive_setup_actions.png}
    \captionof{figure}{Aktionen}
    \label{fig:efalivesetup_actions}
\end{minipage}
\bigskip

Unter Aktionen gibt es aktuell zwei Knöpfe. Einen, um den Computer
herunterzufahren, mit dem anderen kann der Computer neu gestartet
werden. Beide Aktionen müssen noch einmal in einem Dialog bestätigt
werden, damit der Computer nicht unbeabsichtigt heruntergefahren wird.


\subsection{System}
\label{sct:efalivesetup_system}

\begin{minipage}{\linewidth}
    \centering
    \captionsetup{type=figure}
    \includegraphics[width=8cm]{screenshots/efalive_setup_system.png}
    \captionof{figure}{System}
    \label{fig:efalivesetup_system}
\end{minipage}
\bigskip

\subsubsection{Bildschirm-Setup}
\label{sct:bildschirm}
Dieses Werkzeug kann verwendet werden, um den oder die angeschlossenen
Bildschirme zu konfigurieren.

\begin{minipage}{\linewidth}
    \centering
    \captionsetup{type=figure}
    \includegraphics[width=8cm]{screenshots/efaLivede-img22.jpg}
    \captionof{figure}{Bildschirm-Setup Werkzeug}
    \label{fig:bildschirm}
\end{minipage}
\bigskip

Alle erkannten Bildschirme werden mit Namen versehen als graue Rechtecke
in dem Fenster dargestellt. Klickt man mit der rechten Maustaste auf
einen der Bildschirme, öffnet sich ein Kontextmenü, mit dem man den
betreffenden Bildschirm ein- oder ausschalten, rotieren oder die für
diesen Bildschirm verwendete Auflösung verändern kann. Klickt man mit
der linken Maustaste auf einen Bildschirm und hält die Maustaste
gedrückt, kann man die Bildschirme verschieben. Dadurch kann man
mehrere angeschlossene Geräte das gleiche Bild anzeigen lassen, oder
auf verschiedene Weisen nebeneinander anordnen.

Ein Klick auf "`Anwenden"' übernimmt die
Einstellungen für die aktive Darstellung. Mit einem Klick auf
"`Ok"' werden die Einstellungen gespeichert
und bei jedem Start von efaLive automatisch geladen.


\subsubsection{Netzwerk}
\label{sct:dialog_netzwerk}
Über diesen Knopf wird ein Programm namens
"`Network-Manager"' gestartet. Hier können
alle Netzwerkverbindungen des Rechners verwaltet werden. Auf ein paar
Spezialitäten gehe ich im Folgenden ein. Weitere Dokumentation zu dem
"`Network-Manager"' gibt es unter \cite{NWM1}.

\bigskip
\begin{minipage}{\linewidth}
    \centering
    \captionsetup{type=figure}
    \includegraphics[width=12cm]{screenshots/network_manager.png}
    \captionof{figure}{Netzwerk-Einstellungen}
    \label{fig:dialog_netzwerk}
\end{minipage}


\paragraph{W-LAN}
\label{sct:wifi}
Um eine W-LAN Verbindung hinzuzufügen, auf den Knopf "`Hinzufügen"' klicken
und "`Funknetzwerk"' auswählen. Einzutragen ist
mindestens die SSID, also der Namen des W-LAN, in den
Sicherheitseinstellungen ist die Verschlüsselung des WLAN auszuwählen,
sowie das zugehörige Passwort einzutragen. Im Passwortfeld muss über das 
Icon auf der rechten Seite "`Passwort für alle Benutzer speichern"' 
ausgewählt werden. Zusätzlich ist die Option "`Alle Benutzer dürfen 
dieses Netzwerk verwenden"' auszuwählen und ggf. kannst Du einen
Haken bei "`Automatisch mit diesem Netzwerk verbinden, wenn es 
verfügbar ist"' setzen.

\bigskip
\begin{minipage}{\linewidth}
    \centering
    \captionsetup{type=figure}
    \includegraphics[width=13cm]{screenshots/network_manager_wifi.png}
    \captionof{figure}{W-LAN Einstellungen}
    \label{fig:wifi}
\end{minipage}
\bigskip

\paragraph{Breitband}
\label{breitband}
Auch Breitbandverbindungen über z.B. UMTS können konfiguriert werden.
Über den Knopf "`Hinzufügen"' kann ein
Assistent gestartet werden, der bei der Einrichtung unterstützt.
Zuletzt muss noch ein Haken bei "`Automatisch mit diesem Netzwerk 
verbinden, wenn es verfügbar ist"' und "`Alle Benutzer dürfen 
dieses Netzwerk verwenden"' gesetzt und ggf. die PIN eingegeben werden.

\bigskip
\textbf{Achtung:} Durch die Verwendung einer Breitbandverbindung können unter
Umständen hohe Kosten entstehen. Wenn Du keinen entsprechenden Tarif
hast, solltest Du diese Funktion nicht nutzen.
\bigskip

\subsection{Hostname}
\label{sct:ddns}
Hinter diesem Knopf verbirgt sich ein Assistent zur Einrichtung eines
dynamischen DNS Dienstes. Ein solcher Dienst sorgt dafür, dass ein
Computer, der über eine Internetverbindung mit wechselnder IP Adresse
verbunden ist, immer mit einen festen Rechnernamen erreichbar ist. Das
ist für den Zugriff auf efa sehr praktisch. Allerdings funktioniert die
Konfiguration in efaLive nur, wenn der Computer direkt mit dem Internet
verbunden ist. Befindet sich noch ein Router dazwischen, muss die
Konfiguration auf diesem durchgeführt werden.


\subsection{Bildschirmschoner}
\label{bildschirmschoner}
Um die Einstellungen für den Bildschirmschoner, bzw. die
Enegiespareinstellungen für den Bildschirm zu konfigurieren, kann
dieses Programm verwendet werden.


\subsection{Energieverwaltung}
\label{sct:power_management}
Hinter dem Knopf "`Energieverwaltung"' verbirgt sich ein kleines Werkzeug, 
mit dem man einstellen kann, nach welcher Ruhezeit sich der Rechner in einen 
Standby-Modus begeben soll. Dies kann abhängig vom Energieversorgungszustand 
für den Rechner und dessen Bildschirm eingestellt werden.


\subsection{Datum und Uhrzeit}
\label{sct:datetime}
Mit diesem Programm kann die Uhrzeit und das Datum für den Computer
eingestellt werden. Alternativ kann das "`Network Time
Protocol"' verwendet werden. Hierbei wird die Zeit über
das Internet synchronisiert. Das funktioniert natürlich nur unter
Verwendung einer Internetverbindung.

\bigskip
\begin{minipage}{\linewidth}
    \centering
    \captionsetup{type=figure}
    \includegraphics[width=7cm]{screenshots/efaLivede-img25.png}
    \captionof{figure}{Einstelungen Datum \& Uhrzeit}
    \label{fig:datetime}
\end{minipage}
\bigskip

\subsection{Tastatur}
\label{sct:tastatur}
Über diesen Knopf kann die verwendete Tastatur konfiguriert werden. Dies
umfasst sowohl die Einstellungen zum Gerät, als auch die
Tastenbelegung.


\subsection{Aufgaben}
\label{sct:efalivesetup_tasks}

Der efaLive Dämon (\ref{sct:efalivedaemon}) kann verschiedene Aufgaben im
Hintergrund nach einem festgelegten Intervall (stündlich, täglich, 
wöchentlich, monatlich) automatisch ausführen. Über den
Knopf "`+'" kann eine Aufgabe hinzugefügt werden. Über "`-'" wird die aktuell 
ausgewählte Aufgabe gelöscht und mit dem Werkzeugknopf können die Einstellugen
zu der aktuell ausgewählten Aufgabe verändert werden.

\bigskip
\begin{minipage}{\linewidth}
    \centering
    \captionsetup{type=figure}
    \includegraphics[width=8cm]{screenshots/efalive_setup_tasks.png}
    \captionof{figure}{Aufgaben}
    \label{fig:efalivesetup_tasks}
\end{minipage}
\bigskip

Es gibt derzeit zwei verschiedene Typen von Aufgaben: Datensicherung und Shell.

\paragraph{Datensicherung}
Für eine Datensicherung müssen die Einstellungen unter E-Mail (\ref{sct:efalivesetup_email})
vollständig sein. Außerdem muss eine Empfängeradresse angegeben werden.

\paragraph{Shell}
Es können auch Programme aufgerufen werden. Der Befehl muss mit komplettem Pfad in 
das entsprechende Feld eingetragen werden. Zu beachten ist, dass das Kommando nicht 
in einer Shell ausgeführt wird. Umleitungen von Ausgaben per >> funktionieren also
zum Beipiel nicht. Es können aber Shellskripte aufgerufen werden, in denen 
die normalen Shellbefehle funktionieren.


\subsection{Efa Einstellungen}
\label{sct:efalivesetup_efa_settings}

\begin{figure}
    \centering
    \captionsetup{type=figure}
    \includegraphics[width=8cm]{screenshots/efalive_setup_efa-settings.png}
    \caption{Efa Einstellungen}
    \label{fig:efalivesetup_efasettings}
\end{figure}

Die Standardeinstellung 3834 für den Netzwerk Port muss normalerweise nicht 
geändert werden. Nur wenn in efa ein
anderer Port eingestellt wurde, muss der Wert hier entsprechend
angepasst werden.

Die "`Aktion beim Beenden von efa"' regelt das
Verhalten von efaLive, wenn efa beendet wird. Normalerweise wird der
Computer heruntergefahren. Dies ist jedoch nicht immer gewünscht. Man
kann hier stattdessen einstellen, dass der Computer neu gestartet wird, 
dass nur efa neu gestartet wird oder dass efa beendet wird und ein 
grafischer Login erscheint.


\subsection{Datensicherung}
\label{sct:efalivesetup_backup}

\bigskip
\begin{minipage}{\linewidth}
    \centering
    \captionsetup{type=figure}
    \includegraphics[width=8cm]{screenshots/efalive_setup_backup.png}
    \captionof{figure}{Datensicherung}
    \label{fig:efalivesetup_backup}
\end{minipage}
\bigskip

In der Standardeinstellung ist die automatische Datensicherung auf
USB-Sticks abgeschaltet. Durch das Setzen eines Hakens bei
"`Automatische Datensicherung auf USB
einschalten"' kann diese eingeschaltet werden. Ist die
Option aktiv, kann ausgewählt werden, ob nach der automatischen
Datensicherung zusätzlich zu den Tonsignalen ein Dialog angezeigt wird.
Dies ist z.B. nützlich, wenn der Computer keinen Lautsprecher besitzt. 
Außerdem kann ein Passwort eingestellt werden, was nach dem Einstecken
eines USB Sticks abgefragt wird, bevor eine Datensicherung gestartet wird.

Bitte beachte, dass mit eingeschalteter automatischer Datensicherung im
Prinzip jeder eine Datensicherung machen kann, der einen USB Stick in
den Computer stecken kann. In Konfigurationsdateien etc. sind womöglich
sensible Passwörter gespeichert.


\subsection{E-Mail}
\label{sct:efalivesetup_email}

\bigskip
\begin{minipage}{\linewidth}
    \centering
    \captionsetup{type=figure}
    \includegraphics[width=8cm]{screenshots/efalive_setup_email.png}
    \captionof{figure}{E-Mail}
    \label{fig:efalivesetup_email}
\end{minipage}
\bigskip

Hier können Einstellungen für das von efaLive zu verwendene E-Mail-Konto
vorgenommen werden. Das E-Mail-Konto wird für alle zu versendenden E-Mails
von efaLive verwendet. Die Felder sollten weitestgehend elbsterklärend sein.
Das Feld "`Abender"' ist mit der E-Mail-Adresse des gewünschten Absenders zu
füllen.


\section{efaLive-Daemon}
\label{sct:efalivedaemon}
efaLive bringt einen kleinen Service mit, der bestimmte
Aufgaben im Hintergrund erledigt, so dass der Benutzer sich nicht
darum kümmern muss.


\subsection{Watchdog}
\label{sct:watchdog}
Der Watchdog (Wachhund) prüft in regelmäßigen Abständen, ob das System
noch läuft. Konkret prüft das Programm, ob der Fenstermanager "`OpenBox"'
noch aktiv ist. Ist das nicht der Fall, wird der Rechner nach einer gewissen
Zeit neu gestartet. Bitte nicht davon verwirren lasen, dass gegebenfalls
zwischendurch eine Login-Maske sichtbar wird. Hier muss nichts getan werden,
der Rechner wird trotzdem neu gestartet.
Dieser Dienst ist hilfreich, wenn z.B. der X-Server, also das grafische
System von Linux, anstürzt.


\subsection{USB Überwachung}
\label{sct:usb_monitor}
Dieser Dienst überwacht die USB Schnittstellen des Computers und startet
eine Datensicherung, falls das angeschlossene Gerät ein USB Speicherstick
ist und die entsprechende Option in efaLive-Setup (Kapiel \ref{sct:efalivesetup})
eingeschaltet ist.


\section{Software verwalten}
\label{sct:software_verwalten}
\subsection{efa aktualisieren}
\label{update_efa}
Wenn eine aktuellere Version von efa zum Einsatz kommen soll, ist es
nicht nötig, das komplette System neu zu installieren. Es kann die in
efa eingebaute Funktion zur Aktualisierung verwendet werden. Eine 
Aktualisierung ist auch über das Paketmanagement von Linux möglich (siehe
nächster Abschnitt).

Oder man lädt das aktuelle efa von der efa Internetseite \cite{EFA1}
herunter und kopiert es auf einen USB Stick. Diesen USB Stick nun in
einen freien USB Steckplatz des efaLive System einstecken. 

Nun muss man den USB Stick über die Toolbox einbinden und sich wie in
Kapitel \ref{sct:lokaler_zugang} beschrieben auf einer Textkonsole als Benutzer
"`efa"' einloggen und die folgenden Befehle eingeben:
\bigskip
\\
\texttt{cd /usr/lib/efa2\\
unzip -o /media/{\textless}EINHÄNGEPUNKT{\textgreater}/{\textless}NAME\_DER\_EFA\_DATEI{\textgreater}}

\bigskip
\texttt{{\textless}EINHÄNGEPUNKT{\textgreater}} ist hier durch den Namen des USB
Sticks im \texttt{/media/} Verzeichnis zu ersetzen und \texttt{{\textless}NAME\_DER\_EFA\_DATEI{\textgreater}} 
durch den Namen der heruntergeladenen Datei (z.B. \texttt{efa2.zip}).

Nach einem Neustart des Systems sollte automatisch die neue Version von
efa gestartet werden.


\subsection{Linux Software verwalten}
\label{sct:linux_software}
Um weitere Linux Programme zu installieren oder die installierte
Software zu aktualisieren, gibt es im Wesentlichen
drei Verfahren. 

Zum einen können Pakete manuell heruntergeladen und installiert werden,
zum anderen kann man Software von Debian CDs installieren. Außerdem
gibt es die Möglichkeit, das efaLive System so zu konfigurieren, dass
es sich auf Anweisung selbst die entsprechende Software herunterlädt.
Dazu muss jedoch von dem System aus das Internet erreichbar sein.

Die Version der hier verwendeten Debian Distribution ist "`Jessie"'.


\subsubsection{Software manuell installieren}
\label{sct:software_manuell}
Es können unter \cite{DEB3} Softwarepakete für das Live-System
heruntergeladen werden. Diese Softwarepakete haben die Endung
\texttt{.deb} und können auf dem efaLive System
über das Programm \texttt{dpkg} installiert werden. Problematisch bei der
manuellen Installation ist, dass es zwischen den einzelnen Paketen
Abhängigkeiten gibt. Wenn also Programm X installiert werden soll, so
benötigt dieses evtl. noch Programm Y. Auf der Internetseite werden
diese Abhängigkeiten zwar angezeigt, man weiß jedoch nicht unbedingt,
welche der Abhängigkeiten bereits installiert sind. Daher ist dieses
Vorgehen nur für kleine Programme zu empfehlen.

An dieser Stelle funktioniert die Vorgehensweise wie in \ref{sct:software_management}
beschrieben nicht. Hat man eines oder mehrere Pakete heruntergeladen,
werden diese als Benutzer "`root"' mit dem
Befehl \texttt{dpkg -i {\textless}SOFTWARE\_PAKET\_1{\textgreater}
{\textless}SOFTWARE\_PAKET\_2{\textgreater} ...} installiert. 


\subsubsection{Software von CDs}
\label{sct:software_cd}
Unter \cite{DEB4} können CD-Abbilder von der kompletten Debian
Distribution heruntergeladen werden. So steht eine riesige Auswahl an
Software auch ohne Internetzugang zur Verfügung. Je nach Anforderungen
müssen nicht alle CDs heruntergeladen werden. Liegen eine oder mehrere
Debian CDs vor, können diese als Benutzer
"`root"' mit dem Befehl
\texttt{apt-cdrom add} dem System bekannt gemacht
werden. Das Programm fordert automatisch zum Einlegen von CDs auf.

Diese Methode ist vorzuziehen, wenn kein Internetzugang für das efaLive
System zur Verfügung steht.


\subsubsection{Software direkt aus dem Internet}
\label{sct:software_internet}
Steht ein Internetzugang zur Verfügung, so ist dies der komfortabelste
Weg, Software zu installieren oder aktualisieren. Bestand bereits zum
Zeitpunkt der Installation von efaLive ein Internetzugang, so wurde
wahrscheinlich bereits ein Spiegelserver eingerichtet. In diesem Fall fehlt
eventuell noch der Eintrag für das efaLive repository. Sonst müssen als
Benutzer "`root"' die folgenden Befehle ausgeführt werden:
\bigskip
\\
\texttt{echo "deb http://ftp.de.debian.org/debian/ jessie main contrib \textbackslash\\
    non-free"\ {\textgreater}{\textgreater} /etc/apt/sources.list}
\\
\texttt{echo "deb-src http://ftp.de.debian.org/debian/ jessie main contrib \textbackslash\\
    non-free"\ {\textgreater}{\textgreater} /etc/apt/sources.list}
\\
\texttt{echo "deb http://efalive.hannay.de/debian/ jessie main" \textbackslash\\
    {\textgreater}{\textgreater} /etc/apt/sources.list}

\bigskip
Danach muss der interne Index aktualisiert werden. Dies geschieht über
\texttt{apt update}.


\subsubsection{Installieren/Löschen/Suchen/Aktualisieren}
\label{sct:software_management}
Für die Verwaltung der
Linux-Software kann der Befehl \texttt{apt}
benutzt werden. Mit \texttt{apt search
{\textless}STICHWORT{\textgreater}} kann nach Paketen
gesucht werden (funktioniert leider nicht immer sehr gut). Um ein Paket
zu installieren genügt ein \texttt{apt install
{\textless}PAKETNAME{\textgreater}}, um eines zu löschen
\texttt{apt purge
{\textless}PAKETNAME{\textgreater}}. Gegebenenfalls fragt
das Programm apt nach, ob z.B. bestimmte Abhängigkeiten
automatisch mit installiert werden sollen.

Soll die installierte Software aktualisiert werden, kann
\texttt{apt upgrade} aufgerufen werden.

In jedem Fall sollte vor der Nutzung einer dieser Funktionen ein
\texttt{apt update} durchgeführt werden.


\section{Absichern des Systems}
\label{sct:system_absichern}
Es empfiehlt sich, den Computer, der
ja wahrscheinlich im Bootshaus steht und für viele Menschen
zugänglich ist, ein wenig abzusichern. Daher hier ein paar Tipps, wie
man etwas mehr Sicherheit erreichen kann. Allerdings bieten auch all
diese Hinweise keine absolute Sicherheit. Wer sich gut mit Computern
auskennt, wird auch diese Hürden überwinden können. Es ist trotzdem
nützlich, die Latte möglichst hoch zu legen.


\subsection{Peripherie}
\label{sct:peripherie}
Grundsätzlich sollte man darüber nachdenken, ob der Rechner nicht am
Besten in einem abschließbaren Kasten untergebracht werden kann, um die 
Zugangsmöglichkeiten zum System einzuschränken. Ansonsten sollte man 
aus dem Computer alle Hardware ausbauen, die
nicht für den Betrieb benötigt wird. Hier eine Liste von Dingen, die
man oft ausbauen kann:

\begin{itemize}
    \item Diskettenlaufwerke
    \item Netzwerkkarte
    \item Soundkarte
    \item Karten mit seriellen, parallelen oder sonstigen nicht benötigten
        Schnittstellen
    \item CD-ROM Laufwerk (nach der Installation wird es normalerweise nicht
        mehr benötigt)
\end{itemize}


\subsection{BIOS}
\label{sct:bios}
Alles, was nicht physikalisch aus dem Computer ausgebaut werden kann,
aber für den Betrieb von efaLive nicht von Nöten ist, sollte wenigstens
im BIOS des Computers ausgeschaltet werden. Oft gibt es hier die
Möglichkeit, die im Abschnitt \ref{sct:peripherie} erwähnten Geräte abzuschalten.
Außerdem kann man meistens das Starten von Disketten, CDs, USB Sticks
usw. abschalten.

Es empfiehlt sich ferner, ein Passwort für das BIOS zu setzen, damit
Unbefugte die gemachten Einstellungen nicht einfach verändern können.

Manche Computer besitzen einen Schalter im inneren des Gehäuses, der
erkennt, ob das Computergehäuse geöffnet wurde und in einem solchen
Fall für den Start des Computers ein Passwort verlangen. Falls der
verwendete Computer über eine solche Funktion verfügt, bietet es sich
an, diese einzuschalten.


\subsection{Passwörter der Benutzer}
\label{sct:password_admin}
Das Standard-Passwort für den Benutzer "`root"' lautet nach der Installation
"`livecd"'. Es sollte unbedingt geändert werden. Dazu wie unter \ref{sct:lokaler_zugang} beschrieben als
"`root"' mit dem Passwort "`livecd"' einloggen. Das Passwort wird mit
dem Befehl \texttt{passwd} geändert. Das neue
Passwort muss zwei Mal eingegeben werden, um Tippfehlern vorzubeugen. 

Bitte nicht davon verwirren lassen, dass bei der Eingabe von Passwörtern
keinerlei Reaktion auf dem Bildschirm sichtbar wird. Das ist so
gewollt. Erst nach der Bestätigung des Passwortes mit der
\texttt{{\textless}Enter{\textgreater}} Taste, erfolgen wieder Ausgaben auf dem
Bildschirm.

Das Passwort des Benutzers "`efa"', Standard ist "`efalive"', sollte ebenfalls geändert werden. Dazu 
als Benutzer "`efa"' einloggen und, wie oben, den Befehl \texttt{passwd} verwenden.

Das Passwort sollte aus Sicherheitsgründen möglichst lang sein und
Groß-, Kleinbuchstaben, Zahlen und Sonderzeichen enthalten.


\subsection{Passwort Bootloader Grub}
\label{sct:passwort_grub}
Der Auswahlbildschirm des Bootloaders Grub \cite{GRB1} bietet dem
Benutzer viele Möglichkeiten, den Start des Systems zu beeinflussen.
Daher sollte auch hier das voreingestellte Passwort geändert werden.

Dazu, wie in Kapitel~\ref{sct:editor} beschrieben, mit
einem Editor die Datei \linebreak[4]\texttt{/etc/grub.d/40\_custom} editieren. Hier das
voreingestellte Passwort "`livecd"' in der
Zeile \texttt{password root livecd} gegen ein
Eigenes austauschen.


\section{Weiterführende Themen}
\label{sct:weitere_themen}
\subsection{Editor}
\label{sct:editor}
efaLive bringt verschiedene Editoren
mit. Der komfortabelste Editor ist wohl der, der über efaLive-Setup
gestartet werden kann (Kapitel~\ref{sct:gui_editor}).
Wird dieser Editor über efaLive-Setup gestartet, arbeitet man als
Benutzer "`efa"'. Will man Dateien im System
bearbeiten, auf die der Benutzer "`efa"'
keinen Schreibzugriff hat, muss man den Umweg über die Kommandozeile
der Toolbox gehen. Dazu die Kommandozeile starten und mit
\texttt{su -} zum Benutzer
"`root"' wechseln. Nun kann der Editor mit
dem Befehl \texttt{leafpad} gestartet werden.

Es gibt noch zwei Editoren für die Konsole. Zum einen wird mit efaLive
der Editor \texttt{vim} installiert, der zwar sehr
mächtig, aber auch komplizierter von der Bedienung her ist. Daher werde
ich ihn hier nicht näher erklären. Zum anderen gibt es den Editor
\texttt{nano}, den ich hier kurz erläutern will.
Eine Datei kann mit nano editiert werden, indem man z.B.
\texttt{nano /etc/cron.daily/email\_backup}
eingibt, oder auch \texttt{nano email\_backup},
wenn man sich schon in dem Verzeichnis /etc/cron.daily befindet. Wenn
der Editor geöffnet ist, werden am unteren Bildschirmrand verschiedene
Befehle angezeigt. \texttt{\^{}X} z.B. beendet den
Editor. Die Angabe bedeutet, dass zum Beenden die Tastenkombination
\texttt{{\textless}Strg{\textgreater}+{\textless}x{\textgreater}} gedrückt
werden muss.

Hat man nun eine Datei verändert, so kann man zum Speichern
\texttt{{\textless}Strg{\textgreater}+{\textless}O{\textgreater}} drücken oder
gleich \texttt{{\textless}Strg{\textgreater}+{\textless}X{\textgreater}}, da
beim Beenden noch einmal nachgefragt wird, ob die veränderte Datei
gespeichert werden soll (\texttt{{\textless}j{\textgreater}}) oder nicht
(\texttt{{\textless}n{\textgreater}}). In jedem Fall wird nach dem Namen für die
zu speichernde Datei gefragt. Dieser kann für die oben angegebenen
Beispiele einfach bestätigt werden.


\subsection{Kontinuierliche Datensicherung}
\label{sct:kont_backup}
\subsubsection{Auf einen Datenträger}
\label{sct:cont_device}
Um regelmäßig eine Datensicherung auf einem Datenträger zu erzeugen, 
kann eine "`Shell"'-Aufgabe in efaLive-Setup eingerichtet werden. Ist 
das Ziel der Datesicherung ein eingehängtes Dateisystem, dann lautet 
das Kommando \texttt{efalive-backup /{\textless}PFAD{\textgreater}}, wobei 
\texttt{{\textless}PFAD{\textgreater}} durch den Pfad ersetzt werden muss, in dem die 
Datensicherung gespeichert werden soll.

Handelt es sich bei dem Ziel um z.B. einen USB-Stick, muss dieser vor
der Datensicherung eingehängt werden. In diesem Fall lautet das Kommando \\
\texttt{/usr/lib/efalive/bin/autobackup -q /dev/{\textless}DEVICE{\textgreater}}, wobei 
\texttt{{\textless}DEVICE{\textgreater}} durch den richtigen Namen der Gerätedatei ersetzt werden 
muss. Dieser läßt sich mit dem Werkzeug "`Speichermedien"' aus efaLive-Setup ermitteln 
(siehe \ref{sct:dialog_speichermedien}).

Man sollte sich allerdings auf diese Art der Datensicherung nicht
alleine verlassen. Viele Ereignisse, wie z.B. ein Bitzschlag, die die
Festplatte des PCs beschädigen, können gleichzeitig auch den USB Stick
beschädigen. Daher sollte zusätzlich immer noch eine Datensicherung
nach Kapitel \ref{sct:datensicherung} durchgeführt werden.


\subsubsection{Via E-Mail}
\label{cont_mail}
Du kannst eine "`E-Mail Datensicherung"'-Aufgabe in efaLive-Setup erstellen.


\section{Hilfe}
\label{sct:hilfe}
\subsection{Hilfe zu efaLive und efa}
\label{sct:hilfe_efa}
Eine gute Anlaufstelle für Hilfe zu efa und efaLive ist das offizielle
Forum unter \cite{EFA3}. Außerdem gibt es auf der Homepage von efa und
efaLive (\cite{EFA1}\cite{EFA4}\cite{EFA5}) die Dokumentation zu efa
und viele weitere Informationen.


\subsection{Hilfe zu Linux}
\label{sct:hilfe_linux}
Wenn Fragen zu dem Linux-System aufkommen, kann ich nur empfehlen, das
Internet zu nutzen. Über geschickte Anfragen an eine Suchmaschine kann
man zu fast jedem Thema geeignete Hilfen finden. Speziell für Debian
(die Linux Distribution, die efaLive zugrunde liegt) gibt es ein gutes
deutschsprachiges Forum unter \cite{HLP1}. Bevor man jedoch Fragen in
einem solchen Forum stellt, sollte man versuchen, sich selbst mit
bereits im Internet vorhandenen Artikeln oder Foreneinträgen zu helfen.
Schließlich bringt Linux auch Bordmittel zur Hilfe mit. Die sogenannten
Man-Pages geben Auskunft über Befehle und deren Optionen. Für das zur
Installation verwendete Programm \texttt{apt}
kann beispielsweise \texttt{man apt} auf der
Kommandozeile eingegeben werden.

Weitere Informationen kann man auf den folgenden Seiten finden:

\begin{itemize}
    \item \cite{HLP2} - Die häufig gestellten Fragen zu Debian
    \item \cite{HLP3} - Das offizielle Debian Handbuch
\end{itemize}

Zu guter Letzt gibt es natürlich auch viele Bücher zu dem Thema. Wer
allerdings einfach nur ein efa System im Bootshaus aufsetzen möchte,
sollte auch ohne Buch auskommen können.


\clearpage
\section{Anhang}
\label{sct:anhang}
\subsection{Literaturverzeichnis}
\label{sct:literatur}
\bibliographystyle{abbrv}
\bibliography{efaLive_de}


\subsection{Informationen über das System}
\label{sct:sysinfo}

\begin{itemize}
    \item Debian GNU/Linux "`Jessie"' Version 8.5.0
    \item efa Version 2.2.2\_14
\end{itemize}

\end{document}

